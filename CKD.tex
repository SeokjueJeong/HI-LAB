\documentclass{article}

% Language setting
% Replace `english' with e.g. `spanish' to change the document language
\usepackage[english]{babel}
\usepackage{kotex}
% Set page size and margins
% Replace `letterpaper' with `a4paper' for UK/EU standard size
% \usepackage[letterpaper,top=2cm,bottom=2cm,left=3cm,right=3cm,marginparwidth=1.75cm]{geometry}
\usepackage[utf8]{inputenc}
\usepackage[a4paper, total={6in, 10in}]{geometry}
\usepackage{indentfirst}
\usepackage{authblk}
\usepackage{lineno}
\usepackage{natbib}
\usepackage{graphicx}
\usepackage{subcaption}
\usepackage{comment}
% Useful packages
\usepackage{amsmath}
\usepackage{graphicx}
\usepackage[colorlinks=true, allcolors=blue]{hyperref}
\usepackage{color}

\title{관성 센서를 기반으로 딥 러닝을 활용한 CKD 환자 분류}
\author[1,2]{Seok-jue Jeong}
\author[1,2]{Jeonghwan Koh}
\author[2]{Sunghan Lee}
\author[1,2]{Semin Ryu}
\author[1,2]{In cheol Jeong}

\affil[1]{Department of Artificial Intelligence Conversions, Hallym University, Chuncheon, Gangwon, Republic of Korea}
\affil[2]{Cerebrovascular Disease Research Center, Hallym University, Chuncheon, Gangwon, Republic of Korea}


\begin{document}
\maketitle

\begin{abstract}


gait disturbance is impaired in older adults with severe chronic kidney disease (CKD).
최근 증가하는 노인 인구 수에 따른 신장 질환의 문제성이 두드러지고 있다.
만성 신장 질환(CKD)의 단계가 심화된 노인들에게는 공통적으로 보행 장애가
높은 확률로 존재하며, 낙상률 또한 비질환군 보다 높은 추세이다. 그에 따른
결과로 입원 증가 및 건강 관리의 문제가 중요성을 띠게 되면서 CKD 환자의
초기 진단 및 모니터링의 필요성이 커지고 있다. 우리의 연구 목적은 이러한
CKD 질환을 분류하기 위해 의학적 장비 혹은 임상 진단의 방법이 아닌
일반적인 센서를 통해 CKD 환자들의 보행특성을 조사하고 이를 일반인들과
비교했을 때, CKD 중증도의 단계를 구별해낼 수 있는지를 평가하고 강조하는
것이다. 한림대학교 병원 측에서 받은 CKD 중증도 수준이 다른 208명과
동일한 수의 정상군을 통해 실험을 진행하였으며, 2가지의 보행 실험을 통해
연구를 진행하였다. 피실험자들의 보행을 분석하기 위해 손목 및 허리에 각
1개의 IMU 센서를 부착하여 실험을 진행시켰으며, 이를 통해 얻어낸 1D 보행
데이터를 10초 단위로 자르고, 약 0.25초씩 슬라이딩하여 데이터의 양을 대폭
늘렸다. 모델 학습 과정에서는 합성곱 신경망(CNN)및 LSTM 사용하여 CKD
보행을 분류하는 모델로 훈련시켰다. 손목에서 얻어낸 데이터를 가지고
학습을 한 모델에서는 각 실험에서 평균 73.06\%, 70.40\% 정확도를
산출하였고, 허리 데이터를 학습한 모델은 86.97\%, 83.25\%의 정확도를
얻어냈다. 분류 오류의 예상 원인은 데이터의 슬라이딩으로 인한 약간의
데이터 훼손과 입력 데이터로 사용한 일반 1D 원시 데이터의 특징 부족으로
확인이 되었다. 하지만 이 연구를 통해 얻은 결과는 IMU센서를 통해 보행을
분석하고 딥러닝 모델을 통해 CKD의 유무를 확인할 수 있다는 증거를 제공한다.
\end{abstract}

\section{Introduction}

만성신장질환(CKD)은 사구체여과율(GFR)의 감소나 신장의 구조 혹은 기능의
이상이 신체에 문제를 발생시키며 3개월 이상 이 현상이 지속되었을 경우를
말한다. \cite{inker_kdoqi_2014} CKD는 그 중증도에 따라 1단계부터 5단계까지로
이루어져 있으며 그 단계가 높을수록 사구체 여과율이 낮음을 의미한다.
\cite{levey_kdoqi_2002} 신장질환은 원인을 몰라도 진단이 가능하다. Work
Group에 따르면 지속적인 단백뇨는 신장 손상의 주요 지표이다.
\cite{keane_proteinuria_1999} CKD의 잠재적인 결과는 말기 신장 질환(ESRD)으로, 투석또는 이식 형태의 고가의 신장 대체 요법이 필요하다. \cite{zemp_gait_2019}
CKD 3기는 일반 인구, 특히 70세 미만의 인구에 비해 심부전 및 사망의
발생률 증가와 관련이 있다. \cite{eriksen_progression_2006} 이처럼 진행된 CKD단계에서
보행 장애가 생겨 낙상률이 높은 경우가 대다수다. 만성 신장 질환에 의한
하지 근육의 이상 발생으로 인해 보행 장애가 생기기도 한다.
\cite{bohannon_muscle_1994,gordon_association_2012} 그 결과 낙상에 의한 입원 증가, 노인 인구 증가에
따른 보행 장애 증가로 건강 관리에 대한 더 큰 필요성이 요구된다. 이러한
점의 원인이 되는 CKD 단계에서 CKD 환자의 보행 특성을 조사하고 이를 먼저
인식하여 대처할 수 있는 연구의 중요성이 더욱 강조되고 있다.

낙상은 노인 외상의 흔한 원인 중 하나이다. 30세 이상의 지역 사회 거주자
중 65\% 이상이 1년 동안 적어도 한 번 넘어지고, 몸이 약하거나 장애가 있는
사람에게서 낙상 발생률이 더 높다. \cite{speechley_falls_1991} 최근 연구에
따르면, 만성 콩팥병이 있는 65세 이상의 성인은 만성 콩팥병이 없는 같은
연령대의 성인에 비해 낙상과 낙상 부상의 위험이 더 높았다.
\cite{kistler_falls_2018} 이는 확실히 이 집단에서 CKD환자에게서 더 높은
낙상률을 보인다는 것을 설명할 수 있다. 또한 병원에 입원 및 사망한 CKD
환자들 대상으로 연구 결과 대부분 고관절 골절 발생률이 일반 환자들에 비해
높음을 확인할 수 있었다.\cite{robertson_hip_2018}

만성 신장 질환(CKD)은 전 세계 인구의 8\%에서 16\%사이에 영향을 미치며
환자와 임상의가 인식하지 못할 때가 많다.\cite{jha_chronic_2013} 초기 CKD
환자의 5\% 미만이 자신의 질병에 대한 인식을
보고한다.\cite{chen_chronic_2019} 만성 신장 질환(CKD)은 일반적으로 혈청 화학
프로파일 및 소변 검사를 통해 선별 검사를 하고 이를 통한 발견을 통해
확인할 수 있다. 만성 신장 질환의 초기 단계는 실험실에서의 검사를 통해
감지할 수 있다. 만성 신장 질환의 초기 단계의 치료는 신부전으로의 진행을
늦추는 데 효과적이다.\cite{levey_kdoqi_2002} 그러나 보통 CKD는 당뇨병,
심부전증 등 부수적인 질병의 원인이 되는 질병으로 신체의 불편함을 느끼고
입원하였을 때는 병의 진행이 많이 되어있을 경우가 만연하다. 그렇기 때문에
CKD의 초기 단계에서 이를 발견하고 조치를 취하기에는 한계를 가지고 있다.

노인들의 낙상 위험에 대해 평가하기 위해 그들의 보행을 분석하고 특성을
추출하는 연구는 활발히 이용되고 있다.\cite{merlo_postural_2012} CKD와
관련해서도 보행 분석을 진행한 문헌에서 보행과 CKD가 밀접한 관계가 있음을
문헌에서 확인할 수 있다. 일부 문헌에서는 보행 속도의 변화가 만성 신장
질환의 존재에 영향을 받는다는 가설을 세우고 이를 입증할 수 있다고
말한다.\cite{bohannon_deficits_1995,painter_physical_2013,abe_determinants_2016} 하지만 이 부분이 CKD가 아닌 당뇨병에
의해 생기는 문제라는 의견이 존재하기도 한다.\cite{jin_comparison_2017} 확실한
것은 CKD 환자의 보행 특성과 장애의 진행에 따른 신체의 문제점의 연관성을
통해 CKD 집단의 낙상 위험도를 정당화할 수 있고, 이를 통해 CKD의
유무까지도 모니터링 하는데 도움을 줄 수 있다.

근래의 연구에서는 보통 사람의 보행을 분석할 때 용도에 따라 모션 인식
카메라 혹은 웨어러블 센서 등을 이용하여 진행한다.\cite{tao_gait_2012}
보행을 통해 질환의 유무를 판단하려면 미세한 떨림과 자세의 불안정성
정도를 확인하여 결과를 도출해내곤 하는데, 이러한 과정을 거칠 수 있으려면
매우 고가의 카메라 혹은 장비가 요구된다. 힘 판을 이용한다면 이러한
고가의 장비 없이 보행 패턴을 확인할 수 있지만, 힘 판은 허벅지, 팔, 허리
등 보행하였을 때 영향을 받는 부위의 운동학적 정보는 얻어낼 수 없기
때문에 기계 학습을 통해 이를 분류하면 부정확한 결과를 내보일 가능성이
있다.\cite{rashid_gait_2019} 관성 측정 센서(IMU)는 보행을 분석하는 연구를
하는데 접근성이 용이하며 다양한 모듈들과 함께 이용하여 보행 신호를
추출하고 파악하는데 좋은 퍼포먼스를 보이고 있어 많은 주목을 받고 있다.

관성 측정 장치(IMU)라고도 하는 관성 센서는 자체 1차원 로컬 좌표계에서
가속도, 각속도 및 자기장 벡터를 측정한다.\cite{noauthor_sensors_nodate} 이
시스템은 웨어러블 센서로 활용하기 매우 작고 저렴하다. 그 결과
스마트폰에서 이용하거나 웨어러블 센서로 사용하곤
하는데\cite{ngo_largest_2014,ngo_similar_2015}, 많은 연구자들은 사용자
인증\cite{trung_phase_2011}, 보행 분석\cite{tao_gait_2012,rashid_gait_2019,noauthor_sensors_nodate,ngo_largest_2014,ngo_similar_2015}낙상
감지\cite{noury_fall_2007}, 환자의 재활 치료\cite{cifuentes_development_2012} 및 스포츠
훈련 지원\cite{cheng_analysis_2008} 등 다양한 분야에서 관성 센서가 사용되고
있다. 관성 센서의 특징 중 가속도계와 각속도 센서를 통해 보행 속도, 보행
주기 및 자세 유지, 발 각도 등의 여러 보행 패턴 변화를 감지할 수 있어
분석에 용이한 도구로 이용할 수 있다.\cite{caramia_imu-based_2018,dehzangi_imu-based_2017,kim_identification_2021}일부 문헌 중
근육 감소증을 식별하는 연구 과정에서 보행 주기를 신호를 통해 확인하였고
XAI기술을 통해 보행 매개 변수를 추출하여 이용하였다.\cite{kim_identification_2021}
또 HD, PS환자들을 건강한 일반인과 분류하기 위해 진행한 실험에서는 IMU
센서를 5개의 해부학적 위치에 부착하여 보행을
진행하였다.\cite{mannini_machine_2016} 우리 실험은 CKD의 유무를 두고 보행 특성
추출 없이 원시 데이터를 이용하였을 때도 활용가능한 정확도를 산출할 수
있다는 가설을 가지고 임하였다.

관성 센서를 이용하여 실험을 진행하고 인간의 보행 패턴을 찾아내는데 직선
코스로 걸어가는 실험을 했을 때 가장 좋은 정확도를
산출하였다.\cite{bohannon_deficits_1995,painter_physical_2013,jin_comparison_2017,caramia_imu-based_2018,dehzangi_imu-based_2017,kim_identification_2021}11개의 연구에서
피실험자들은 일어나 준비된 상태로 시작을 하였고, 피실험자들이 자신이
선택한 속도로 보행을 진행하게끔
하였다.\cite{bohannon_deficits_1995,abe_determinants_2016,painter_physical_2013,jin_comparison_2017,tao_gait_2012,rashid_gait_2019,ngo_similar_2015,caramia_imu-based_2018,dehzangi_imu-based_2017,kim_identification_2021,mannini_machine_2016}또한 최근 연구에서, 웨어러블
센서의 부착 위치로 관성 측정 센서를 운동학적 정보에 따라
5\textasciitilde7개의 부위에 부착하였을 때 가장 좋은 정확도를
산출하였다. \cite{caramia_imu-based_2018,mannini_machine_2016}

보행 분석을 진행하는데 여러 기계학습 방법들이 좋은 성과를 보이고 있다.
특히 CNN을 이용하여 그 layer의 깊이를 더 깊게 하거나 다른 모델과 섞어서
쓰는 등 보행 분석 및 분류에서 그 활용 가치는 매우
높다.\cite{rashid_gait_2019,dehzangi_imu-based_2017,kim_identification_2021,mannini_machine_2016,eskofier_recent_2016,wang_gait_2020} Eskofier\cite{eskofier_recent_2016}는
AdaBoost, PART, kNN, SVM 및 컨볼루션 신경망(CNN)의 적용을 기반으로 PD를
분류했으며 CNN이 가장 높은 정확도를 산출했다. Kim\cite{kim_identification_2021}은
SVM, 랜덤 포레스트, 다층 퍼셉트론 및 딥 러닝 방법을 적용했으며
근감소증을 분류하는데 SVM이 가장 높은 정확도를 산출하였다. 또다른 일부
문헌에서는 CNN 학습을 통해 걷기, 서기, 앉기, 눕기, 조깅과 같은 동작을
분류하였다.\cite{dehzangi_imu-based_2017}

본 연구에서는 다양한 분류 방법 중 딥 러닝 기법을 통해 CKD 환자의 보행
특성을 학습하여 높은 정확도로 분류하는 알고리즘을 제안한다. 관성 센서를
통해 평소 타인의 시선 혹은, 자신까지도 인지하기 힘든 보행장애를 그
사람의 손목과 허리에서의 각속도 및 가속도를 측정해, 이를 원시 데이터로
추출하여 사용하였다. 우리는 받은 데이터를 입력으로 CNN을 통해 CKD의
단계가 다양한 환자군과 정상 보행자를 학습시켜 새로운 입력이 들어왔을 때
이들을 각 class마다 높은 정확도로 구분해낼 수 있다고 가설을 내리고 이의
타당성을 검증하였다.

\section{Material and Method}

\subsection{장비 수집}

관성측정센서(IMU)는 일반적으로 가속도계와 자이로스코프 센서로 구성된
장치로 소비자의 필요에 따라 자력계 센서도 포함된다. 우리는 실험 과정에서
MPU-9250 센서를 이용하였고, 총 9축의 자유도 센서로 구성되있다. 실험 도중
자력계 센서의 데이터는 보행 신호 분석에 미미한 영향을 끼침이 확인되어
이를 제외한 총 6축의 데이터만을 분석에 활용하였다.

IMU 센서의 가속도계는 \(\pm\)16g 및 2048 LSB/g의 가속도 범위에서
가속도를 측정할 수 있다. 자이로스코프 센서를 이용하였을 때는
\(\pm 500\)g 범위에서 회전 속도를 측정할 수 있다. 센서의 각 축 방향과
회전 각은 그림1에서 확인할 수 있다.

\begin{figure}
\centering
\includegraphics[width=3.86944in,height=2.92153in]{Fig.1.png}
\caption{\label{Figure.1:}Axes and Angles of Accelerometer and Gyroscope Sensors}
\end{figure}

\subsection{데이터 수집}

한림성심병원에서 체지방 및 환자의 검체 검사 결과를 근거로 만성 콩팥병의
척도 및 유무를 확인하여 실험군을 모집하였다. 개인의 CKD 중증도를
다양하게 보유하고 있는 성인 남녀 208명과 정상인 208명으로부터 보행
데이터를 수집하였다. 이들 모두의 통계 조사를 실시했으며 자가심리 평가와
만족도 조사 등 동의 하에 진행하였다.

416명의 피험자들은 두 가지 실험에 임했다. 첫 실험에서는 피 실험자들은
직선으로 (그림3)10걸음을 걷는 실험에 임하였고, 이를 두 번 진행하였다. 두
번째 실험에서는 (그림2)3M 타원 보행을 진행하였으며 이를 6번 반복하였다.
이후 다른 섹션에서의 이해를 돕기 위해 첫 실험을 10s 보행이라는 이름으로,
두 번째 실험을 ST 실험이라고 정하겠다. 모든 실험에서 피험자들은 왼쪽
손목과 허리에 관성 센서의 부착을 진행하였고 데이터의 문제가 발생했을 시
그 데이터는 과감하게 폐기하였다. 실험에서의 사람들은 각각 보행 속도에서
약간의 차이를 보였지만 다중 분류의 문제가 될 정도의 데이터는 존재하지 않았다.

\begin{figure}
\centering
\includegraphics[width=3.86944in,height=2.92153in]{Fig.2.png}
\caption{\label{Figure.2:}Elliptical Walking: The subject walks straight for 3 meters, makes a turn at the junction, and returns for another 3 meters.}
\end{figure}

\begin{figure}
\centering
\includegraphics[width=4.71319in,height=1.89583in]{Fig.3.png}
\caption{\label{Figure.3:}10-Step Straight Walking: The subject walks straight for 10 steps and then stops.}
\end{figure}

최종적으로 두 가지 실험에서 허리와 손목의 센서로부터 보행 데이터를
얻어냈다. 관성 센서의 샘플링 속도는 50Hz에서 측정되었다.

\subsection{보행 주기 식별}

우리는 원시 데이터의 형태만을 통해 학습을 시켰기 때문에 절대적으로 원시
데이터의 가치가 높아야 한다. 그렇기 때문에 데이터에 반드시 필요한
정보로서 보행의 일정한 주기가 내포되어 있어야만 한다. Misu
등\cite{misu_development_2017}은 힐 스트라이크(HS)를 가속도 신호의 피크 지점
주기를, 토오프(TO)는 각속도 신호의 피크지점 주기를 기반으로 보행 주기를
검출하였다. 우리는 원시 데이터의 2D 이미지를 이용해 그래프의 한 주기를
보행 주기로 인식하였고, 입력 데이터로 들어가는 한 샘플에 이 주기가
반드시 들어갈 수 있게끔 시간을 조정해 데이터를 형성하였다. (그림4)

\begin{figure}
\centering
\includegraphics[width=6.26806in,height=2.63264in]{Fig.4.png}
\caption{\label{Figure.4:}Input Size for Each Experiment}
\end{figure}

\subsection{데이터 개수 늘리기}

우리는 두 실험에서 표본의 길이, 즉 보행 주기를 뽑기 위해 보는 시간을
달리 하여 진행하였는데, 10S 실험은 10걸음의 신호만 담겨있는 데이터기
때문에 ST실험보다 짧은 데이터로 같은 시간을 보기엔 데이터가 터무니없이
적었다. 
(이 부분에 표 혹은 그래프를 활용하여 100,300,500로 샘플링 rate를
맞추었을 때의 샘플 개수를 나타내어 10S에서는 300Hz를 활용한 것에 대한
당위성을 제공) 확인해보시고 피드백 해주시면 바로 수정하겠습니다.

그래서 10S 실험에서는 300Hz의 Window size를 입력 데이터로 활용하기로
하였고, ST 실험에서는 500Hz의 크기만큼 활용하였다. 각각 6초 및 10초의
시간을 하나의 입력으로 하였을 때 데이터의 정확도는 높일 수 있으나 데이터
양의 유한성으로 결과의 과적합을 불러일으킬 수 있는 상황이었다. 따라서
이를 해결하고자 각 입력으로 들어가는 데이터를 약 0.25초씩 뒤로 슬라이딩
하여 그림5의 형태대로 데이터의 개수를 무작위로 늘릴 수 있도록 하였다.

\begin{figure}
\centering
\includegraphics[width=6.26806in,height=1.67083in]{Fig.5.png}
\caption{\label{Figure.5:}To augment data, apply window sliding in increments of 12 samples.}
\end{figure}

\subsection{보행 데이터 추출}

만성 콩팥병 환자 208명과 정상 보행자 208명이 두 가지 실험에 참여하여
각각 2번과 6번씩 진행하였고, 각 실험마다 다른 입력 데이터를 추가로
12samples씩 슬라이딩 하여 손목과 허리의 데이터를 얻었다. 그림6에서
보듯이 손목과 허리의 변환 가능한 데이터의 개수가 서로 다른데 이는 각
데이터에서 미씽 현상이 발생하거나 샘플링이 하나씩 밀려 사용할 수 없는
데이터를 각 부위에서 제외하였더니 발생한 현상이다. 우리는 실험에서 두
가지의 부위에서의 발생하는 특징을 사용할 것이기에 더 작은 양의 데이터에
맞추어 많은 양의 데이터가 나오는 부위에서 무작위로 데이터를 폐기하여
개수를 맞추었다. 관성 센서의 가속도계 3축, 자이로스코프 3축만을 사용하여
이들을 분석하였고 모든 데이터는 각 열의 피처 값의 평균을 0으로 잡고
표준편차를 1로 간주하여 정규화를 진행하였다.

\begin{figure}
\centering
\includegraphics[width=6.26806in,height=3.15417in]{Fig.6.png}
\caption{\label{Figure.6:}Available pedestrian data by body part}
\end{figure}

이렇게 해서 얻은 데이터를 각 실험에 입력하여 진행시키기 위해 두 실험
모두 훈련 데이터와 테스트 데이터로 나누기 위해 먼저 개수를 약 4:1의
비율로 나누었다. 이후 각 데이터의 표본편향을 방지하기 위해 5-Fold 로
교차 검증을 진행하였다.

\begin{figure}
\centering
\includegraphics[width=6.20903in,height=3.8in]{Fig.7.png}
\caption{\label{Figure.7:}In the experiment, k-Fold cross validation was performed with a 4:1 Train-Test ratio.}
\end{figure}

그 결과 데이터를 5개로 나누어 각 부분을 훈련 및 검증에 사용하기위해 각각
나누었다. ST 실험에서는 허리 부위에서 평균 Train set 17000개, Test set
약 5000개의 4:1 비율로 나누어 실험을 진행하였고, 손목 부위에서는 약
28000개, 5400개 정도로 나누어 진행하였다. 10S 실험에서는 ST 실험보다
전체적으로 개수가 적기 때문에 허리 부위에선 평균 Train set 2300개, Test
set 700개의 정도의 개수 비율로 진행하였고, 손목 부위 또한 Train set
3400개, Test set 600개 정도로 진행하였다. 각 Fold 당 조금씩 개수 차이는
존재하였지만, 그 파일 개수가 10개 미만 차이가 나고 4:1의 비율을
유지하였기 때문에 Fold 당 큰 성능 차이는 보이지 않았다.

\begin{figure}
\centering
\includegraphics[width=6.26806in,height=4.17847in]{Fig.8.png}
\caption{\label{Figure.8:}Experiment-specific data distribution}
\end{figure}

\subsection{기계학습}

우리는 관성 센서의 신호와 원시 데이터를 사용하고 특징 추출의 기계학습을
제외한 방법 중 CKD를 식별하는 방법을 탐구했다. CNN-BiLSTM은 감정 분석,
자연어 처리, 신호 분석 등에서 유용하게 사용되는 딥러닝 방법으로 각각의
최상의 결과를 산출해내는데 많은 호응을 받고 있다.

CNN은 이미지의 특징을 추출하는 부분과 클래스를 분류하는 부분으로 나눌 수
있다. 그 중 특징 추출 영역은 하나 이상의 Convolution Layer와 Pooling
Layer를 여러 층 쌓는 형태로 구성된다. 이후 분류 영역에서 Fully-connected
신경망에서 입력으로 받은 특징을 바탕으로 분류한다. CNN은 피처맵을
생성하는 필터까지도 학습이 가능해 비전 분야에서 우수한 평가를 받고 있다.

LSTM은 순차 데이터를 처리하는 데 효과적인 순환 신경망(RNN)의 한 유형으로
전통적인 RNN과는 다르게 LSTM 유닛은 시퀀스 내에서 장거리 종속성을
포착하고 그라디언트 소실 문제를 완화하는 데 효과적으로 작동한다. LSTM의
구조는 시계열 분석, 자연어 처리, 음성 인식과 같은 순차 데이터를 다루는
작업에 적합하여 적당한 조건 속에서 활용되고 있다.

우리 연구에서는 1D-CNN을 활용하여 센서 데이터에 따른 공간적 특징에서의
종속성을 확인한 뒤 여기다 Bi-LSTM 레이어를 통합함으로써 순차 정보를
효과적으로 포착하고 이것이 중요한 순차 정보를 제공하여 연구 목적에 맞는
결과를 증명해낼 수 있도록 모델을 쌓아 진행하였다.

자세히 살펴보면, 6축의 관성 센서의 데이터 (500,2)의 shape를 모델의
입력으로 이용하였고, 3번의 Convolution layer를 쌓았으며 각각 Batch
Normalization을 통해 정규화를 하여 나타냈다. 이후 평균 풀링을 통해 값의
크기를 반으로 줄여주었다. 이 다음 얻어낸 특징과 BI-LSTM을 통합하여
최종적 특징 추출을 거친 다음 Dense 층을 활용하여 모델의 깊이를 더해갔고,
드롭아웃을 주어 과적합을 방지하였다. 분류기로는 다중 분류에 특화된
softmax를 활용하였다. 모델의 자세한 형태는 그림 7에서 확인할 수 있다.
\begin{figure}
\centering
\includegraphics[width=5.09583in,height=3.71319in]{Fig.9.1.png}
\includegraphics[width=6.26806in,height=1.69931in]{Fig.9.2.png}
\caption{\label{Figure.9:}The model architeture based on CNN-LSTM.}
\end{figure}

\section{Result}

손목 및 허리 부위의 두 가지 실험에서 모두 같은 파라미터와 모델을
활용하였고, 모델의 Epoch 또한 300번으로 통일하여 진행하였다. 10s 실험은
ST 보행 실험보다 데이터의 총량이 월등히 적기 때문에 원시 데이터만을
이용하는 두 가지 실험에선 10S 실험의 정확도가 낮은 부분을 확인할 수
있었다. 실험 결과는 CKD의 유무와 함께 중증도 별 정확도를 나타내었다.

\subsection{10s 실험}

10s 실험은 인 당 두 번씩 진행하였으며 CNN모델을 통해 정확도 및 여러
결과들을 시각화 하여 산출하였다.

300번의 Epoch을 주어 정확도를 산출해본 결과, 손목 부위에서 Best
accuracy로 75.6\%의 정확도를 얻었고 허리 부위에서는 92.47\%의 정확도를
얻어내었다. 혼동행렬을 통해 확인해본 결과 허리 부위는 매우 분류가 잘
되는 것으로 확인이 되나 손목 부위에서는 CKD 환자들을 잘 구분하지 못하는
것을 확인할 수 있었다.

\begin{table}
\centering
\begin{tabular}{l|r}
Point &  & Precision & Recall & F1-score & Support \\\hline
Hand & \\
Fold1 & 
\end{tabular}
\caption{\label{tab:widgets}An example table.}
\end{table}

\begin{figure}
\centering
\includegraphics[width=6.26806in,height=3.00069in]{Fig.13.png}
\caption{\label{TABLE 2:}EXPERIMENT RESULTS: BINARY CLASSIFICATION REPORT FOR WAIST POINT IN 10S EXPERIMENT}
\end{figure}

\begin{figure}
\centering
\includegraphics[]{Fig.10.png}
\caption{\label{Figure.10:}In the 10S experiment, normal and CKD patient groups were used as input data, and the confusion matrix represented the proportions for the wrist (left) and waist (right) regions.}
\end{figure}

추가로 정상 보행자 따로, Class1과 2를 묶고, Class 3,4 마지막으로 5와 6을
묶어 각 4집단으로 나누어 분류를 진행시켰다. 이를 통해 중증도가 비슷한
집단간에도 확연히 차이가 나는지에 대한 가설을 해결하고자 하였다.

먼저 300번의 Epoch을 주어 정확도를 산출해본 결과, 손목 부위에서 Best
accuracy로 75\%의 정확도를 얻었고 허리 부위에서는 75\%의 정확도를
얻어내었다.

표5,6. 허리와 손목의 10s 실험 4 Class Classification 결과(미완)

\subsection{ST 보행 실험}

ST 보행 실험은 인 당 6번씩 진행하였고, 위와 같은 파라미터 및 형식으로
실험을 진행하였다.

여러 번의 시도 중 가장 좋은 결과를 도출하였을 때, 손목에 센서를 부착하여
시도하였을 때는 약 75.22\%의 정확도를 산출하였고, 허리에서 시도하였을 때
약 90.43\%의 결과를 산출하였다. ST 보행 실험 결과 관련하여 Confusion
matrix를 뽑은 결과와 그에 대한 세부적인 결과 내용은 표4과 그림 9에서
확인할 수 있다.

\begin{figure}
\centering
\includegraphics[width=6.26806in,height=3.00069in]{Fig.14.png}
\caption{\label{TABLE 3:}EXPERIMENT RESULTS: BINARY CLASSIFICATION REPORT FOR HAND POINT IN ST EXPERIMENT}
\end{figure}

\begin{figure}
\centering
\includegraphics[width=6.26806in,height=3.00069in]{Fig.15.png}
\caption{\label{TABLE 4:}EXPERIMENT RESULTS: BINARY CLASSIFICATION REPORT FOR WAIST POINT IN ST EXPERIMENT}
\end{figure}

\begin{figure}
\centering
\includegraphics[width=6.26806in,height=2.97986in]{Fig.11.png}
\caption{\label{Figure.11:}In the ST experiment, normal and CKD patient groups were used as input data, and the confusion matrix represented the proportions for the wrist (left) and waist (right) regions.}
\end{figure}

ST 실험에서도 마찬가지로 Multi class classification을 진행하여 결과를
확인하였다.

먼저 500번의 Epoch을 주어 정확도를 산출해본 결과, 손목 부위에서 Best
accuracy로 75\%의 정확도를 얻었고 허리 부위에서는 75\%의 정확도를
얻어내었다.

표9,10. 허리와 손목 부위의 ST 실험 4 Class Classification 결과(미완)

\section{Discussion}

모델의 입력으로 사용한 원시 데이터 같은 경우 다른 통계적, 서술적 보행 파라미터 없이 그대로 이용할 경우 정확도가 뛰어나게 나오기 어렵다.
Kim{[}15{]}은 근감소증 환자를 정상인과 판별하는데 원시 데이터만을
이용하여 약 54\%의 정확도를 도출하였다. 만약 보행 파라미터를 검출하여
주기를 정확하게 계산하고 이를 여러 분류 알고리즘에 대입한다면 매우
뛰어난 결과가 나올 수 있는 분류 문제임은 어느정도 예상이 가능했지만,
우리는 이 과정에서 드는 비용과 노력을 거치지 않고도 가능케하였다.

원시 데이터 중 보행 주기를 입력 데이터에 완벽히 존재하려면 데이터의 양이
매우 적어지고, 이를 슬라이딩으로 밀어 데이터의 수를 증가시키면 보행
주기가 완벽히 들어가지 못한 데이터가 형성된다. 그렇기 때문에 우리는
윈도우 슬라이딩을 통해 데이터의 개수를 늘리되 최대한 모든 데이터의
가치를 잃어버리지 않게끔 시간을 조정하였다. 학습할 데이터에 측정된
관성센서 기록의 timestamp가 있어야 누락 데이터가 존재할 시
interpolation이 가능한데 이 부분이 존재하지 않아 우리는 누락 데이터를
제외시키고 실험을 진행하였다. 하지만 만약 두 부위를 같이 사용하는 Multi
modal model을 만들어 이용하려면 데이터 타임의 시작과 끝이 일치해야
학습이 가능하기 때문에 이는 해결이 필요한 부분이다.

최종적으로 Bi-LSTM을 이용하였을 때 최상의 결과가 나왔기 때문에 이 모델을
채택하였지만, 아직 진행해보지 못한 모델들도 여럿 있었고 원시
데이터이기에 시계열 데이터에 효과적인 모델들 모두 비슷한 결과를 나타내
보였다. 특이하게도 10S 실험이 훨씬 더 적은 데이터 양을 가지고 있음에도
불구하고 비슷한 성능 결과를 나타낸 것을 확인한 것으로 보아 양 차이 보다
확실히 그 데이터의 질, 즉 허리 부위에서 추출한 데이터가 더 분류해내는데
의미 있는 내용을 내포하고 있음을 알 수 있었다. 그럼에도 불구하고 보행
변수를 따로 추출해 분류하는 과정이 아니기 때문에 원시 데이터만을 이용할
때는 최대한 많은 양의 보행 과정을 추출하는 것이 좋은 결과를 산출해내는데
많은 도움이 될 것으로 확인된다.

하지만 결과에서 나온 것처럼 손목 부위에서의 정확도는 평균 73\%정도로
분류에서 좋은 성능이 나타났다고 하기 부족한 부분이 있다. 우리는 이
부분을 보아 보행에서 나타나는 팔의 스윙은 CKD 환자와 정상 환자를
구별해내는 데에는 충분한 근거가 내포되어 있지 않음을 확인하였고, 하지
부분의 영향성에 더 집중할 수 있도록 하였다.

보행 주기를 입력으로 활용할 때 보행을 하는데 확실한 기점이 될 수 있는
주기를 스텝 별로 확인하여 이를 데이터로 입력하였다면 확실히 데이터의
가치성이 높아지고, 데이터의 수도 월등히 많아질 수 있기 때문에 비록
이번엔 보행을 한 그 시계열 데이터를 그대로 이용하여 확인했지만 전자의
방법으로 시도하여 더 좋은 성능을 얻어낼 수 있을 것이라고 기대된다.

CKD의 모든 단계의 사람 인원 수가 같지 않고 각각 많고 적은 단계가
형성되어 있었기 때문에 비교적 CKD 0기로 오인하여 결과를 도출할 경우가
많아 정확도에 악영향을 끼친 경우도 있었다. 이는 각 중증도의 인원수를
맞게 하여 진행한다면 각 클래스 별 특성에 따라 구별이 가능한 지를 더 의미
있게 찾을 수 있을 것으로 확인된다.

\section{Conclusion}

인간의 보행을 인식하고 분류하는데 중요한 목표로는 두 보행 종류에서
차이점을 찾아내어 이를 분석하여 학습시키는 것이다. 현재 만성
콩팥병이라는 질환의 진단은 병원에서의 정밀 검사를 통해 확인할 수 있는
상태이다. 우리는 CKD 환자의 보행은 특이점이 존재하여 이들을 병원의
고가의 장비 없이도 먼저 확인하고 진단할 수 있다고 가설을 두고 이 실험을
진행 및 평가하였다.

일상 생활에서 CKD로 의한 증상이 발현되기 전 혹은 발현되어 의심될 때
생기는 보행의 특징을 읽기 위해서는 실용적이여야 하고 일반인들도
접근하기에 쉬운 특성을 띠고 있어야만 한다. 우리는 이러한 특성을 생각하여
IMU 센서라는 이미 많은 보행 분석 과정에서 사용되어 그 유용함을 입증 받고
있는 기기를 이용하였다. 이 실험에서 특히 관성 센서의 신호에는 보행에
대해 분석 및 예측하는데 충분한 정보와 효율성이 있다는 것을 보여주었다.
두 가지의 실험에서 우리는 원시 데이터만을 이용하여 딥 러닝 기법을 통해
최상의 결과를 산출하였다. 원시 데이터만을 이용했을 때 보행 변수를
사용했을 때보다 정확도는 낮아도 그만큼 고가의 장비 혹은 복잡한 특징 추출
과정이 없기 때문에 그 실용성은 충분하다. 우리는 데이터를 CKD 환자와 정상
군들을 먼저 나누어 모델에 학습을 시킨 뒤 진행하였고, CKD 환자의 중증도
별로 Class를 나누어 정상 군을 포함한 4개의 군집으로 편성하여 분류를
시도하였다. 최종 결과로는 새로운 데이터셋을 이 모델을 검증하는데
이용하였다. 그 과정 속에서 2가지 부위에서의 정확도를 확인하였고, 입력
데이터를 제외하고 다른 부분들은 통일하여 Multi Class classification을
진행하였을 때 정확도의 차이를 확인하였다. 모델의 선택 과정에서 학습하는
데의 시간과 분류하는 데의 정확도를 중점으로 선택하게 되었고, 그 결과
CNN-BiLSTM이 가장 좋은 결과를 도출하였다. 이미 CNN-Bi LSTM 기법은 여러
분류 기법 쪽에서도 뛰어난 실용성과 유용함이 입증되었기 때문에 CKD를
판별하는 과정에서도 사용하는 충분한 이유를 보일 수 있었다. 우리는 이를
이용하여 ST 실험에서 허리 부위에서 평균 약 87\%의 정확도를 산출하여 CKD
환자의 보행을 분류하는데 충분한 근거를 산출해냈으나 손목에서의 정확도는
관계성을 증명하기에 미약한 부분이 있음을 확인하였다.

따라서 허리 벨트 등의 의료기기를 이용하여 일상생활에서도 만성 콩팥병의
유무 및 단계를 확인하고 진단할 수 있다는 점을 확신하였다. 향후에는
추가적으로 같은 원시 데이터를 받지만 두 부위 혹은 더 많은 부위를 이용한
멀티 모달 모델을 만들어 이를 이용할 수 있다면 CKD 단계를 분류하는데 가장
핵심이 되는 부위를 알아내거나 단계에 따른 걸음걸이에서 특이점을 확인할
수 있을 것으로 기대된다.

\bibliography{related}

\section{Reference}

1. Inker, L. A. \emph{et al.} KDOQI US Commentary on the 2012 KDIGO
Clinical Practice Guideline for the Evaluation and Management of CKD.
\emph{Am. J. Kidney Dis.} \textbf{63}, 713--735 (2014).

2. Levey, A. S. \emph{et al.} K/DOQI clinical practice guidelines for
chronic kidney disease: Evaluation, classification, and stratification.
\emph{Am. J. Kidney Dis.} \textbf{39}, i-S266 (2002).

3. Keane, W. F. \& Eknoyan, G. Proteinuria, albuminuria, risk,
assessment, detection, elimination (PARADE): A position paper of the
National Kidney Foundation. \emph{Am. J. Kidney Dis.} \textbf{33},
1004--1010 (1999).

4. Zemp, D. D., Giannini, O., Quadri, P. \& de Bruin, E. D. Gait
characteristics of CKD patients: a systematic review. \emph{BMC
Nephrol.} \textbf{20}, 83 (2019).

5. Eriksen, B. O. \& Ingebretsen, O. C. The progression of chronic
kidney disease: A 10-year population-based study of the effects of
gender and age. \emph{Kidney Int.} \textbf{69}, 375--382 (2006).

6. Gordon, P. L., Doyle, J. W. \& Johansen, K. L. Association of
1,25-Dihydroxyvitamin D Levels With Physical Performance and Thigh
Muscle Cross-sectional Area in Chronic Kidney Disease Stage 3 and 4.
\emph{J. Ren. Nutr.} \textbf{22}, 423--433 (2012).

7. Bohannon, R. W., Hull, D. \& Palmeri, D. Muscle Strength Impairments
and Gait Performance Deficits in Kidney Transplantation Candidates.
\emph{Am. J. Kidney Dis.} \textbf{24}, 480--485 (1994).

8. Speechley, M. \& Tinetti, M. Falls and injuries in frail and vigorous
community elderly persons. \emph{J. Am. Geriatr. Soc.} \textbf{39},
46--52 (1991).

9. Kistler, B. M., Khubchandani, J., Jakubowicz, G., Wilund, K. \&
Sosnoff, J. Falls and Fall-Related Injuries Among US Adults Aged 65 or
Older With Chronic Kidney Disease. \emph{Prev. Chronic. Dis.}
\textbf{15}, E82 (2018).

10. Robertson, L. \emph{et al.} Hip fracture incidence and mortality in
chronic kidney disease: the GLOMMS-II record linkage cohort study.
\emph{BMJ Open} \textbf{8}, e020312 (2018).

11. Jha, V. \emph{et al.} Chronic kidney disease: global dimension and
perspectives. \emph{The Lancet} \textbf{382}, 260--272 (2013).

12. Chen, T. K., Knicely, D. H. \& Grams, M. E. Chronic Kidney Disease
Diagnosis and Management. \emph{JAMA} \textbf{322}, 1294--1304 (2019).

13. Merlo, A. \emph{et al.} Postural stability and history of falls in
cognitively able older adults: The Canton Ticino study. \emph{Gait
Posture} \textbf{36}, 662--666 (2012).

14. Bohannon, R. W., Smith, J., Hull, D., Palmeri, D. \& Barnhard, R.
Deficits in lower extremity muscle and gait performance among renal
transplant candidates. \emph{Arch. Phys. Med. Rehabil.} \textbf{76},
547--551 (1995).

15. Painter, P. \& Marcus, R. Physical Function and Gait Speed In
Patients with Chronic Kidney Disease. \emph{Nephrol. Nurs. J.}
\textbf{40}, 529--38; quiz 539 (2013).

16. Abe, Y. \emph{et al.} Determinants of Slow Walking Speed in
Ambulatory Patients Undergoing Maintenance Hemodialysis. \emph{PloS One}
\textbf{11}, e0151037 (2016).

17. Jin, S. H., Park, Y. S., Park, Y. H., Chang, H. J. \& Kim, S. R.
Comparison of Gait Speed and Peripheral Nerve Function Between Chronic
Kidney Disease Patients With and Without Diabetes. \emph{Ann. Rehabil.
Med.} \textbf{41}, 72--79 (2017).

18. Tao, W., Liu, T., Zheng, R. \& Feng, H. Gait Analysis Using Wearable
Sensors. \emph{Sensors} \textbf{12}, 2255--2283 (2012).

19. Rashid, U., Kumari, N., Taylor, D., David, T. \& Signal, N. Gait
Event Anomaly Detection and Correction During a Split-Belt Treadmill
Task. \emph{IEEE Access} \textbf{PP}, (2019).

20. Sensors \textbar{} Free Full-Text \textbar{} IMU-Based Joint Angle
Measurement for Gait Analysis. https://www.mdpi.com/1424-8220/14/4/6891.

21. Ngo, T. T., Makihara, Y., Nagahara, H., Mukaigawa, Y. \& Yagi, Y.
The largest inertial sensor-based gait database and performance
evaluation of gait-based personal authentication. \emph{Pattern
Recognit.} \textbf{47}, 228--237 (2014).

22. Ngo, T. T., Makihara, Y., Nagahara, H., Mukaigawa, Y. \& Yagi, Y.
Similar gait action recognition using an inertial sensor. \emph{Pattern
Recognit.} \textbf{48}, 1289--1301 (2015).

23. Trung, N. T. \emph{et al.} Phase registration in a gallery improving
gait authentication. in \emph{2011 International Joint Conference on
Biometrics (IJCB)} 1--7 (2011). doi:10.1109/IJCB.2011.6117527.

24. Noury, N. \emph{et al.} Fall detection - Principles and Methods. in
\emph{2007 29th Annual International Conference of the IEEE Engineering
in Medicine and Biology Society} 1663--1666 (2007).
doi:10.1109/IEMBS.2007.4352627.

25. Cifuentes, C. \emph{et al.} Development of a wearable ZigBee sensor
system for upper limb rehabilitation robotics. in \emph{2012 4th IEEE
RAS \& EMBS International Conference on Biomedical Robotics and
Biomechatronics (BioRob)} 1989--1994 (2012).
doi:10.1109/BioRob.2012.6290926.

26. Cheng, L. \& Hailes, S. Analysis of Wireless Inertial Sensing for
Athlete Coaching Support. in \emph{IEEE GLOBECOM 2008 - 2008 IEEE Global
Telecommunications Conference} 1--5 (2008).
doi:10.1109/GLOCOM.2008.ECP.1006.

27. Caramia, C. \emph{et al.} IMU-Based Classification of Parkinson's
Disease From Gait: A Sensitivity Analysis on Sensor Location and Feature
Selection. \emph{IEEE J. Biomed. Health Inform.} \textbf{22}, 1765--1774
(2018).

28. Dehzangi, O., Taherisadr, M. \& ChangalVala, R. IMU-Based Gait
Recognition Using Convolutional Neural Networks and Multi-Sensor Fusion.
\emph{Sensors} \textbf{17}, 2735 (2017).

29. Kim, J.-K., Bae, M.-N., Lee, K. B. \& Hong, S. G. Identification of
Patients with Sarcopenia Using Gait Parameters Based on Inertial
Sensors. \emph{Sensors} \textbf{21}, 1786 (2021).

30. Mannini, A., Trojaniello, D., Cereatti, A. \& Sabatini, A. M. A
Machine Learning Framework for Gait Classification Using Inertial
Sensors: Application to Elderly, Post-Stroke and Huntington's Disease
Patients. \emph{Sensors} \textbf{16}, 134 (2016).

31. Eskofier, B. M. \emph{et al.} Recent machine learning advancements
in sensor-based mobility analysis: Deep learning for Parkinson's disease
assessment. in \emph{2016 38th Annual International Conference of the
IEEE Engineering in Medicine and Biology Society (EMBC)} 655--658
(2016). doi:10.1109/EMBC.2016.7590787.

32. Wang, X. \& Zhang, J. Gait feature extraction and gait
classification using two-branch CNN. \emph{Multimed. Tools Appl.}
\textbf{79}, 2917--2930 (2020).

33. Misu, S. \emph{et al.} Development and validity of methods for the
estimation of temporal gait parameters from heel-attached inertial
sensors in younger and older adults. \emph{Gait Posture} \textbf{57},
295--298 (2017).


\end{document}
